\documentclass{article}
\usepackage[utf8]{inputenc}
\usepackage[utf8]{inputenc}
\usepackage[spanish]{babel}
\title{}
\author{}
\date{}
\pdfinfo{%
  /Title    (Inteligencia Artificial I - Proyecto II)
  /Author   (Fabio Castro, Maria Jorge, Jorge Marcano)
  /Creator  ()
  /Producer ()
  /Subject  ()
  /Keywords ()
}
\begin{document}




 
\title{\Huge Inteligencia Artificial I-Proyecto II Othello }



\author{Fabio Castro 10-10132, María Jorge 11-10495, Jorge Marcano 11-10566} 



\date{08/06/2015}

\maketitle

\section{Introducción}
\hspace{0.5cm}En teoría de juego, un árbol de juego es un grafo dirigido de tipo árbol cuyos nodos representan posiciones en el juego y cuyas aristas representan movimientos de los jugadores. Cualquier sucesión de jugadas puede representarse por un camino conexo dentro del árbol de juego. Si el juego acaba siempre después de un número finito de pasos, entonces el árbol tiene un número finito de nodos.

Sabemos que existen juegos de dos personas en donde cada persona juega por un turno y el objetivo es jugar de la mejor manera posible. Este tipo de juegos se pueden representar con un árbol de decisión, especificamente en uno llamado “And Or”, que está
constituido en cada nivel del árbol solamente por nodos tipo “And” o solamente tiene nodos tipo “Or”. 

La Computación toma su papel en estos juegos a la hora de calcular la mejor manera posible de jugar para cada jugador. En el caso de un juego de ajedrez, podemos calcular la mejor manera de jugar buscando hacer el jaque
mate lo más rápido posible, o simplemente asegurar un jaque mate en una jugada futura. 

En nuestro caso haremos uso del juego Othello, el cual es un juego de estrategia para dos jugadores en un tablero de 6x6 donde cada
jugador por turnos coloca una ficha por su cara blanca o negra según el color que le corresponda a cada jugador. En
el juego cada vez que se coloque una pieza se voltearán todas las piezas adyacentes a ésta
que sean del otro color hasta llegar a una del color de la pieza jugada. Si no hay otra pieza
del mismo color de la pieza a colocar en la fila, columna o alguna de las diagonales de la
posición donde se quiere jugar entonces esa jugada no es posible. Gana el jugador que, al
no quedar espacios libres en el tablero, tenga la mayor cantidad de fichas colocadas. 
 De manera intuitiva también podemos decir que muchas veces es mejor buscar una solución cercana a la óptima, que encontrar la solución óptima, ya que recortamos el espacio de búsqueda. 


El objetivo de este proyecto es probar el desempeño, entre los algoritmos de búsqueda
en grafos conocidos como Negamax, Negamax alpha-beta, Scout y Negascout, para resolver el juego
de Othello de forma tal que ambos jugadores (MIN y MAX) muevan sus piezas de la mejor
forma posible.

\section{Algoritmos}
\hspace{0.5cm}Negamax es un método de decisión para minimizar la pérdida máxima esperada en juegos con adversario. Este cálculo se hace de forma recursiva.
El funcionamiento de Negamax puede resumirse en cómo elegir el movimiento que más te favorezca, suponiendo que tu contrincante escogerá el peor para ti. El
tiempo empleado para explorar el árbol es aproximadamente igual al número de
ramificaciones o posibles jugadas que tenga en el momento cada jugador, exponencial a la
profundidad de la búsqueda.

El problema de la búsqueda Negamax es que el número de estados a explorar es exponencial al número de movimientos. Partiendo de este hecho, la técnica de poda alfa-beta trata de eliminar partes grandes del árbol, aplicándolo a un árbol Negamax estándar, de forma que se devuelva el mismo movimiento que devolvería este, gracias a que la poda de dichas ramas no influye en la decisión final

Negascout es un algoritmo de búsqueda, variante de Minimax, que disminuye el
número de nodos expandidos usando la ventana nula. Este algoritmo mejora los tiempos del
Minimax al no explorar ramas que no darán una mejor solución, esto se sabe gracias a los
valores de alpha y beta.

Por su parte Scout   es   un   algoritmo   que   busca   recortar   el   espacio   de   búsqueda   haciendo   uso   de   una   función  
auxiliar llamada TEST, y recordando los parientes de un nodo del árbol para su beneficio. 

Todos los algoritmos utilizan heurísticas que permite optimizarlos y así obtener buenos
resultados en buenos tiempos de ejecución. En este proyecto utilizaremos el
valor del juego como heurística, es decir, la resta entre fichas que posee el jugador MAX y fichas que posee el
jugador MIN.

\section{Implementación}

\hspace{0.5cm}Para la implementación utilizamos el lenguaje C++. Contamos con el archivo
othello\_cut.h, donde se define la clase state\_t, que representa los estados del tablero de juego. Además
se definen funciones sobre estos necesarias para el funcionamiento de los algoritmos de búsqueda. 

Para la implementación de los algoritmos realizamos un archivo para cada uno de ellos y utilizamos el contenido del archivo main.cc para poder realizar las corridas desde estados de la variación principal, es decir, dependiendo de la profundidad que se deseara probar, avanzábamos en la variación principal hasta llegar al estado "inicial" desde donde empezaría la ejecución de alguno de los algoritmos. Además, la salida por pantalla indica el tiempo de ejecución de la corrida, los nodos generados y los nodos finales alcanzados.

A continuación presentamos tablas comparativas con datos recopilados de la
ejecución de cada algoritmo sobre árboles de juego de Othello de diferentes tamaños. Para cada algoritmo se
presentan los nodos generados, nodos finales alcanzados y el tiempo de ejecución. Todos los algoritmos encontraron el valor del juego
perfecto igual a \-4, donde gana el jugador MIN por 4 puntos. Para la realización de las tablas se tomó como cota de tiempo máximo 10 minutos, por tanto, si una corrida no terminaba en esa duración de tiempo, ninguna otra que estuviese más cerca de la raíz de la variación principal lo haría. Por esta razón tomamos el último nivel en el que se pudo resolver el problema.

\section{Resultados}


\begin{table}[ht]
\begin{tabular}{|l|l|l|l|l|l|l|l|}
\hline
      &                & \multicolumn{3}{c|}{NegaMax}                      & \multicolumn{3}{c|}{NegaMaxAB}                  \\ \hline
nivel & val & \#gen & \#fin & tiempo         & \#gen & \#fin & tiempo         \\ \hline
30    & -4             & 5               & 2             & 0.000012s  & 1               & 1             & 0.000005s  \\ \hline
29    & 4              & 12              & 4             & 0.000028s  & 4               & 2             & 0.000011s  \\ \hline
28    & -4             & 13              & 4             & 0.000031s  & 5               & 2             & 0.000012s  \\ \hline
27    & 4              & 90              & 27            & 0.000180s  & 12              & 4             & 0.000029s  \\ \hline
26    & -4             & 176             & 52            & 0.000350s  & 13              & 4             & 0.000031s  \\ \hline
25    & 4              & 1048            & 305           & 0.002176s  & 26              & 6             & 0.000058s  \\ \hline
24    & -4             & 4497            & 1330          & 0.009451s  & 81              & 20            & 0.000173s  \\ \hline
23    & 4              & 11977           & 3381          & 0.024579s  & 237             & 52            & 0.000534s  \\ \hline
22    & -4             & 76825           & 21699         & 0.163949s  & 1002            & 234           & 0.002261s  \\ \hline
21    & 4              & 428401          & 119923        & 0.908753s  & 1501            & 350           & 0.003440s  \\ \hline
20    & -4             & 3478734         & 953486        & 7.392386s  & 4067            & 900           & 0.009746s  \\ \hline
19    & 4              & 13078932        & 3619363       & 27.637522s & 9129            & 2099          & 0.023140s  \\ \hline
18    & -4             & 90647894        & 25526376      & 189.487928s& 98754           & 22734         & 0.230288s  \\ \hline
17    & 4              & NT              & NT            & NT              & 127643          & 29515         & 0.303847s  \\ \hline
16    & -4             & NT              & NT            & NT              & 267603          & 62587         & 0.629561s  \\ \hline
15    & 4              & NT              & NT            & NT              & 1259429         & 299087        & 2.973716s  \\ \hline
14    & -4             & NT              & NT            & NT              & 2031923         & 482139        & 4.846172s  \\ \hline
13    & 4              & NT              & NT            & NT              & 29501797        & 7176690       & 67.372291s \\ \hline
12    & -4             & NT              & NT            & NT              & 43574642        & 10625624      & 100.456035s\\ \hline
11    & 4              & NT              & NT            & NT              & 107642870       & 25626713      & 247.318001s\\ \hline
10    & -4             & NT              & NT            & NT              & NT              & NT            & NT              \\ \hline
9    & NT             & NT              & NT            & NT              & NT              & NT            & NT              \\ \hline
8    & NT             & NT              & NT            & NT              & NT              & NT            & NT              \\ \hline
\end{tabular}
\end{table}

\begin{table}[ht]
\begin{tabular}{|l|l|l|l|l|l|l|l|}
\hline
      &                & \multicolumn{3}{c|}{Scout}                       & \multicolumn{3}{c|}{NegaScout}                   \\ \hline
nivel & val & \#gen & \#fin & tiempo         & \#gen & \#fin & tiempo         \\ \hline
30    & -4             & 1               & 1             & 0.000006s& 1               & 1             & 0.000006s\\ \hline
29    & 4              & 4               & 2             & 0.000011s& 4               & 2             & 0.000011s\\ \hline
28    & -4             & 5               & 2             & 0.000011s& 5               & 2             & 0.000012s\\ \hline
27    & 4              & 7               & 6             & 0.000040s& 17              & 6             & 0.000040s\\ \hline
26    & -4             & 18              & 6             & 0.000042s& 18              & 6             & 0.000042s\\ \hline
25    & 4              & 31              & 8             & 0.000070s& 31              & 8             & 0.000070s\\ \hline
24    & -4             & 81              & 20            & 0.000179s& 81              & 20            & 0.000169s\\ \hline
23    & 4              & 389             & 88            & 0.000871s& 389             & 86            & 0.000859s\\ \hline
22    & -4             & 1804            & 445           & 0.004049s& 1631            & 393           & 0.003588s\\ \hline
21    & 4              & 2717            & 655           & 0.006005s& 2421            & 571           & 0.005433s\\ \hline
20    & -4             & 4229            & 955           & 0.009825s& 3843            & 847           & 0.008934s\\ \hline
19    & 4              & 14900           & 3544          & 0.033689s& 11979           & 2756          & 0.027654s\\ \hline
18    & -4             & 78268           & 17770         & 0.182093s& 48477           & 10728         & 0.113865s\\ \hline
17    & 4              & 202034          & 46661         & 0.467732s& 81631           & 18326         & 0.190300s\\ \hline
16    & -4             & 314746          & 72717         & 0.731536s& 184144          & 41839         & 0.431409s\\ \hline
15    & 4              & 850441          & 199368        & 1.972837s& 605947          & 140374        & 1.419510s\\ \hline
14    & -4             & 1407448         & 331087        & 3.264474s& 1134288         & 264924        & 2.643214s\\ \hline
13    & 4              & 8367183         & 1995588       & 19.235866s & 7221811         & 1706005       & 16.623680s \\ \hline
12    & -4             & 42907223        & 10308619      & 97.288906s & 25831374        & 6129025       & 59.355238s \\ \hline
11    & 4              & 83581446        & 19674433      & 189.939757s& 62051335        & 14453944      & 141.531664s\\ \hline
10    & -4             & NT              & NT            & NT             & 242584981       & 57293575      & 541.282530s\\ \hline
9    & NT             & NT              & NT            & NT             & NT              & NT            & NT             \\ \hline
8    & NT             & NT              & NT            & NT             & NT              & NT            & NT             \\ \hline
\end{tabular}
\end{table}

\clearpage

\section{Conclusiones}
\hspace{0.5cm}De las tablas presentadas de las ejecuciones de los algoritmos podemos inferir que el algoritmo Negamax es el más ineficiente, esto se debe a que el espacio de búsqueda es demasiado grande, ya que este recorre todo el árbol de juego antes de tomar una decisión. Con la cota de tiempo establecida, fue el algoritmo que menos niveles pudo resolver, llegando al nivel 18.

Lo ideal es usar algoritmos que realicen podas sobre el espacio de búsqueda como lo son los algoritmos Negamax con poda alpha beta, Scout o Negascout, que logran podar significativamente el espacio de
búsqueda.
El mejor algoritmo de los utilizados, fue Negascout, seguido de Scout, NegaMax Alpha beta y por último Negamax.
Finalmente, el secreto de Negascout radica en que asume que el nodo actual que está abierto es el
mejor y que se encuentra en la variación principal, buscando así si es cierto con una ventana nula de
alpha beta, lo que termina siendo mucho más rápido que una 
busqueda con alpha beta normal.


\bibliographystyle{plain}
\bibliography{Referencias}
http://ldc.usb.ve/~bonet/courses/ci5437/othello/ \\
http://www.cs.cornell.edu/~yuli/othello/othello.html \\
http://elvex.ugr.es/decsai/algorithms/slides/problems/Games.pdf \\
http://profesores.elo.utfsm.cl/~tarredondo/info/soft-comp/Arboles\%20de\%20Decision.pdf\\
http://es.wikipedia.org/wiki/Negascout\\

\end{document}
