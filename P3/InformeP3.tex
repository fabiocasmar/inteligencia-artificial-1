\documentclass{article}
\usepackage[utf8]{inputenc}
\usepackage[utf8]{inputenc}
\usepackage[spanish]{babel}
\title{}
\author{}
\date{}
\pdfinfo{%
  /Title    (Inteligencia Artificial I - Proyecto II)
  /Author   (Fabio Castro, Maria Jorge, Jorge Marcano)
  /Creator  ()
  /Producer ()
  /Subject  ()
  /Keywords ()
}
\begin{document}




 
\title{\Huge Inteligencia Artificial I-Proyecto III }



\author{Fabio Castro 10-10132, María Jorge 11-10495, Jorge Marcano 11-10566} 



\date{22/06/2015}

\maketitle

\section{Introducción}
\hspace{0.5cm}

\section{Ejercicios}
\hspace{0.5cm}
\subsection{Ejercicio 1}

\subsection{Ejercicio 2}


Para la implementacion de este ejercicio, decidimos representar el tablero como una matriz de tamano 7x7 de numeros enteros
entre el 1 y el 7. \par


Sobre este tablero, aplicamos las restricciones de que las columnas sean todas distintas entre si, y lo mismo
para las filas. Para hacer esto, utilizamos la funcion alldifferent, la cual, dado un arreglo y un rango de sus
elementos,se garantiza que los elementos dentro de ese rango seran distintos entre si. \par


Para aplicar las restricciones sobre los 7 caminos, creamos 7 arreglos de enteros entre 1 y 7 que representan los caminos, y
verificamos que cada miembro de cada camino sea igual a su posicion en el tablero original, y colocamos otras restricciones
adicionales usando alldifferent para asegurarnos que los elementos de cada camino sean distintos entre si. \par


Al correr el programa con dichas restricciones, obtuvimos el siguiente resultado : \par


(COLOCAR RESULTADO AQUI) \par

\subsection{Ejercicio 3}
\subsection{Ejercicio 4}
\subsection{Ejercicio 5}


En este ejercicio decidimos representar el tablero de juego como una matriz de dimensiones NxM (donde N y M eran la cantidad de filas y columnas indicadas en el enunciado respectivamente). Adicionalmente, representamos cada una de las 12 piezas como otra matriz de dimensiones 5x5, en donde las posiciones correspondientes a la pieza estaban marcadas con un numero del 1 al 12 segun cual pieza fuera, y el resto con 0. \par


Por medio de restricciones, nos aseguramos de que cada pieza se encuentre en una posicion valida, tomando en cuenta como validas, las rotaciones de 90,180 y 270 grados de la pieza original, y las mismas rotaciones,para la reflexion de la misma pieza. \par


Por ultimo, revisamos que ninguna de las piezas colocadas en el tablero se solape con otras. \par


Al ejecutarlo obtuvimos los siguientes resultados : \par


(RESULTADOS AQUI, Y/O JUSTIFICACION SOBRE POR QUE NO NOS DIO Y LO QUE LOGRAMOS HACER).\par


\clearpage

\section{Conclusiones}
\hspace{0.5cm}

\bibliographystyle{plain}
\bibliography{Referencias}


\end{document}
